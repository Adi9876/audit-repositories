\begin{titlepage}
    \centering
    \begin{figure}[h]
        \centering
        \includegraphics[width=0.5\textwidth]{logo.pdf} 
    \end{figure}
    \vspace*{2cm}
    {\Huge\bfseries TSwap Protocol Audit Report\par}
    \vspace{1cm}
    {\Large Version 1.0\par}
    \vspace{2cm}
    {\Large\itshape adiverse\par}
    \vfill
    {\large \today\par}
\end{titlepage}

\maketitle

Prepared by: {[}Aditya{]}

\section{Table of Contents}\label{table-of-contents}

\begin{itemize}
\tightlist
\item
  \hyperref[table-of-contents]{Table of Contents}
\item
  \hyperref[protocol-summary]{Protocol Summary}
\item
  \hyperref[disclaimer]{Disclaimer}
\item
  \hyperref[risk-classification]{Risk Classification}
\item
  \hyperref[audit-details]{Audit Details}

  \begin{itemize}
  \tightlist
  \item
    \hyperref[scope]{Scope}
  \item
    \hyperref[roles]{Roles}
  \end{itemize}
\item
  \hyperref[executive-summary]{Executive Summary}

  \begin{itemize}
  \tightlist
  \item
    \hyperref[issues-found]{Issues found}
  \end{itemize}
\item
  \hyperref[findings]{Findings}

  \begin{itemize}
  \tightlist
  \item
    \hyperref[high]{High}

    \begin{itemize}
    \tightlist
    \item
      \hyperref[h-1-tswappooldeposit-is-missing-deadline-check-causing-transactions-to-complete-even-after-the-deadline]{{[}H-1{]}
      \texttt{TSwapPool::deposit} is missing deadline check causing
      transactions to complete even after the deadline}
    \item
      \hyperref[h-2-incorrect-fee-calculation-in-tswappoolgetinputamountbasedonoutput-causes-protocll-to-take-too-many-tokens-from-users-resulting-in-lost-fees]{{[}H-2{]}
      Incorrect fee calculation in
      \texttt{TSwapPool::getInputAmountBasedOnOutput} causes protocll to
      take too many tokens from users, resulting in lost fees}
    \item
      \hyperref[h-3-lack-of-slippage-protection-in-tswappoolswapexactoutput-causes-users-to-potentially-receive-way-fewer-tokens]{{[}H-3{]}
      Lack of slippage protection in \texttt{TSwapPool::swapExactOutput}
      causes users to potentially receive way fewer tokens}
    \item
      \hyperref[h-4-tswappoolsellpooltokens-mismatches-input-and-output-tokens-causing-users-to-receive-the-incorrect-amount-of-tokens]{{[}H-4{]}
      \texttt{TSwapPool::sellPoolTokens} mismatches input and output
      tokens causing users to receive the incorrect amount of tokens}
    \item
      \hyperref[h-5-in-tswappool_swap-the-extra-tokens-given-to-users-after-every-swapcount-breaks-the-protocol-invariant-of-x--y--k]{{[}H-5{]}
      In \texttt{TSwapPool::\_swap} the extra tokens given to users
      after every \texttt{swapCount} breaks the protocol invariant of
      \texttt{x\ *\ y\ =\ k}}
    \end{itemize}
  \item
    \hyperref[low]{Low}

    \begin{itemize}
    \tightlist
    \item
      \hyperref[l-1-tswappoolliquidityadded-event-has-parameters-out-of-order]{{[}L-1{]}
      \texttt{TSwapPool::LiquidityAdded} event has parameters out of
      order}
    \item
      \hyperref[l-2-default-value-returned-by-tswappoolswapexactinput-results-in-incorrect-return-value-given]{{[}L-2{]}
      Default value returned by \texttt{TSwapPool::swapExactInput}
      results in incorrect return value given}
    \end{itemize}
  \item
    \hyperref[informationals]{Informationals}

    \begin{itemize}
    \tightlist
    \item
      \hyperref[i-1-poolfactorypoolfactory__pooldoesnotexist-is-not-used-and-should-be-removed]{{[}I-1{]}
      \texttt{PoolFactory::PoolFactory\_\_PoolDoesNotExist} is not used
      and should be removed}
    \item
      \hyperref[i-2-lacking-zero-address-checks]{{[}I-2{]} Lacking zero
      address checks}
    \item
      \hyperref[i-3-poolfacotrycreatepool-should-use-symbol-instead-of-name]{{[}I-3{]}
      \texttt{PoolFacotry::createPool} should use \texttt{.symbol()}
      instead of \texttt{.name()}}
    \item
      \hyperref[i-4-event-is-missing-indexed-fields]{{[}I-4{]} Event is
      missing \texttt{indexed} fields}
    \end{itemize}
  \end{itemize}
\end{itemize}

\section{Protocol Summary}\label{protocol-summary}

Protocol does X, Y, Z

\section{Disclaimer}\label{disclaimer}

The YOUR\_NAME\_HERE team makes all effort to find as many
vulnerabilities in the code in the given time period, but holds no
responsibilities for the findings provided in this document. A security
audit by the team is not an endorsement of the underlying business or
product. The audit was time-boxed and the review of the code was solely
on the security aspects of the Solidity implementation of the contracts.

\section{Risk Classification}\label{risk-classification}

\begin{longtable}[]{@{}lllll@{}}
\toprule\noalign{}
& & Impact & & \\
\midrule\noalign{}
\endhead
\bottomrule\noalign{}
\endlastfoot
& & High & Medium & Low \\
& High & H & H/M & M \\
Likelihood & Medium & H/M & M & M/L \\
& Low & M & M/L & L \\
\end{longtable}

We use the
\href{https://docs.codehawks.com/hawks-auditors/how-to-evaluate-a-finding-severity}{CodeHawks}
severity matrix to determine severity. See the documentation for more
details.

\section{Audit Details}\label{audit-details}

\subsection{Scope}\label{scope}

\subsection{Roles}\label{roles}

\section{Executive Summary}\label{executive-summary}

\subsection{Issues found}\label{issues-found}

\begin{longtable}[]{@{}ll@{}}
\toprule\noalign{}
Severtity & Number of issues found \\
\midrule\noalign{}
\endhead
\bottomrule\noalign{}
\endlastfoot
High & 4 \\
Medium & 2 \\
Low & 2 \\
Info & 9 \\
Total & 17 \\
\end{longtable}

\section{Findings}\label{findings}

\subsection{High}\label{high}

\subsubsection{\texorpdfstring{{[}H-1{]} \texttt{TSwapPool::deposit} is
missing deadline check causing transactions to complete even after the
deadline}{{[}H-1{]} TSwapPool::deposit is missing deadline check causing transactions to complete even after the deadline}}\label{h-1-tswappooldeposit-is-missing-deadline-check-causing-transactions-to-complete-even-after-the-deadline}

\textbf{Description:} The \texttt{deposit} function accepts a deadline
parameter, which according to the documentation is ``The deadline for
the transaction to be completed by''. However, this parameter is never
used. As a consequence, operationrs that add liquidity to the pool might
be executed at unexpected times, in market conditions where the deposit
rate is unfavorable.

\textbf{Impact:} Transactions could be sent when market conditions are
unfavorable to deposit, even when adding a deadline parameter.

\textbf{Proof of Concept:} The \texttt{deadline} parameter is unused.

\textbf{Recommended Mitigation:} Consider making the following change to
the function.

\begin{Shaded}
\begin{Highlighting}[]
\NormalTok{function deposit(}
\NormalTok{        uint256 wethToDeposit,}
\NormalTok{        uint256 minimumLiquidityTokensToMint, // LP tokens {-}\textgreater{} if empty, we can pick 100\% (100\% == 17 tokens)}
\NormalTok{        uint256 maximumPoolTokensToDeposit,}
\NormalTok{        uint64 deadline}
\NormalTok{    )}
\NormalTok{        external}
\VariableTok{+       revertIfDeadlinePassed(deadline)}
\NormalTok{        revertIfZero(wethToDeposit)}
\NormalTok{        returns (uint256 liquidityTokensToMint)}
\NormalTok{    \{}
\end{Highlighting}
\end{Shaded}

\subsubsection{\texorpdfstring{{[}H-2{]} Incorrect fee calculation in
\texttt{TSwapPool::getInputAmountBasedOnOutput} causes protocll to take
too many tokens from users, resulting in lost
fees}{{[}H-2{]} Incorrect fee calculation in TSwapPool::getInputAmountBasedOnOutput causes protocll to take too many tokens from users, resulting in lost fees}}\label{h-2-incorrect-fee-calculation-in-tswappoolgetinputamountbasedonoutput-causes-protocll-to-take-too-many-tokens-from-users-resulting-in-lost-fees}

\textbf{Description:} The \texttt{getInputAmountBasedOnOutput} function
is intended to calculate the amount of tokens a user should deposit
given an amount of tokens of output tokens. However, the function
currently miscalculates the resulting amount. When calculating the fee,
it scales the amount by 10\_000 instead of 1\_000.

\textbf{Impact:} Protocol takes more fees than expected from users.

\textbf{Recommended Mitigation:}

\begin{Shaded}
\begin{Highlighting}[]
\NormalTok{    function getInputAmountBasedOnOutput(}
\NormalTok{        uint256 outputAmount,}
\NormalTok{        uint256 inputReserves,}
\NormalTok{        uint256 outputReserves}
\NormalTok{    )}
\NormalTok{        public}
\NormalTok{        pure}
\NormalTok{        revertIfZero(outputAmount)}
\NormalTok{        revertIfZero(outputReserves)}
\NormalTok{        returns (uint256 inputAmount)}
\NormalTok{    \{}
\StringTok{{-}        return ((inputReserves * outputAmount) * 10\_000) / ((outputReserves {-} outputAmount) * 997);}
\VariableTok{+        return ((inputReserves * outputAmount) * 1\_000) / ((outputReserves {-} outputAmount) * 997);}
\NormalTok{    \}}
\end{Highlighting}
\end{Shaded}

\subsubsection{\texorpdfstring{{[}H-3{]} Lack of slippage protection in
\texttt{TSwapPool::swapExactOutput} causes users to potentially receive
way fewer
tokens}{{[}H-3{]} Lack of slippage protection in TSwapPool::swapExactOutput causes users to potentially receive way fewer tokens}}\label{h-3-lack-of-slippage-protection-in-tswappoolswapexactoutput-causes-users-to-potentially-receive-way-fewer-tokens}

\textbf{Description:} The \texttt{swapExactOutput} function does not
include any sort of slippage protection. This function is similar to
what is done in \texttt{TSwapPool::swapExactInput}, where the function
specifies a \texttt{minOutputAmount}, the \texttt{swapExactOutput}
function should specify a \texttt{maxInputAmount}.

\textbf{Impact:} If market conditions change before the transaciton
processes, the user could get a much worse swap.

\textbf{Proof of Concept:} 1. The price of 1 WETH right now is 1,000
USDC 2. User inputs a \texttt{swapExactOutput} looking for 1 WETH 1.
inputToken = USDC 2. outputToken = WETH 3. outputAmount = 1 4. deadline
= whatever 3. The function does not offer a maxInput amount 4. As the
transaction is pending in the mempool, the market changes! And the price
moves HUGE -\textgreater{} 1 WETH is now 10,000 USDC. 10x more than the
user expected 5. The transaction completes, but the user sent the
protocol 10,000 USDC instead of the expected 1,000 USDC

\textbf{Recommended Mitigation:} We should include a
\texttt{maxInputAmount} so the user only has to spend up to a specific
amount, and can predict how much they will spend on the protocol.

\begin{Shaded}
\begin{Highlighting}[]
\NormalTok{    function swapExactOutput(}
\NormalTok{        IERC20 inputToken, }
\VariableTok{+       uint256 maxInputAmount,}
\NormalTok{.}
\NormalTok{.}
\NormalTok{.}
\NormalTok{        inputAmount = getInputAmountBasedOnOutput(outputAmount, inputReserves, outputReserves);}
\VariableTok{+       if(inputAmount \textgreater{} maxInputAmount)\{}
\VariableTok{+           revert();}
\VariableTok{+       \}        }
\NormalTok{        \_swap(inputToken, inputAmount, outputToken, outputAmount);}
\end{Highlighting}
\end{Shaded}

\subsubsection{\texorpdfstring{{[}H-4{]}
\texttt{TSwapPool::sellPoolTokens} mismatches input and output tokens
causing users to receive the incorrect amount of
tokens}{{[}H-4{]} TSwapPool::sellPoolTokens mismatches input and output tokens causing users to receive the incorrect amount of tokens}}\label{h-4-tswappoolsellpooltokens-mismatches-input-and-output-tokens-causing-users-to-receive-the-incorrect-amount-of-tokens}

\textbf{Description:} The \texttt{sellPoolTokens} function is intended
to allow users to easily sell pool tokens and receive WETH in exchange.
Users indicate how many pool tokens they're willing to sell in the
\texttt{poolTokenAmount} parameter. However, the function currently
miscalculaes the swapped amount.

This is due to the fact that the \texttt{swapExactOutput} function is
called, whereas the \texttt{swapExactInput} function is the one that
should be called. Because users specify the exact amount of input
tokens, not output.

\textbf{Impact:} Users will swap the wrong amount of tokens, which is a
severe disruption of protcol functionality.

\textbf{Proof of Concept:}

\textbf{Recommended Mitigation:}

Consider changing the implementation to use \texttt{swapExactInput}
instead of \texttt{swapExactOutput}. Note that this would also require
changing the \texttt{sellPoolTokens} function to accept a new parameter
(ie \texttt{minWethToReceive} to be passed to \texttt{swapExactInput})

\begin{Shaded}
\begin{Highlighting}[]
\NormalTok{    function sellPoolTokens(}
\NormalTok{        uint256 poolTokenAmount,}
\VariableTok{+       uint256 minWethToReceive,    }
\NormalTok{        ) external returns (uint256 wethAmount) \{}
\StringTok{{-}        return swapExactOutput(i\_poolToken, i\_wethToken, poolTokenAmount, uint64(block.timestamp));}
\VariableTok{+        return swapExactInput(i\_poolToken, poolTokenAmount, i\_wethToken, minWethToReceive, uint64(block.timestamp));}
\NormalTok{    \}}
\end{Highlighting}
\end{Shaded}

Additionally, it might be wise to add a deadline to the function, as
there is currently no deadline. (MEV later)

\subsubsection{\texorpdfstring{{[}H-5{]} In \texttt{TSwapPool::\_swap}
the extra tokens given to users after every \texttt{swapCount} breaks
the protocol invariant of
\texttt{x\ *\ y\ =\ k}}{{[}H-5{]} In TSwapPool::\_swap the extra tokens given to users after every swapCount breaks the protocol invariant of x * y = k}}\label{h-5-in-tswappool_swap-the-extra-tokens-given-to-users-after-every-swapcount-breaks-the-protocol-invariant-of-x-y-k}

\textbf{Description:} The protocol follows a strict invariant of
\texttt{x\ *\ y\ =\ k}. Where: - \texttt{x}: The balance of the pool
token - \texttt{y}: The balance of WETH - \texttt{k}: The constant
product of the two balances

This means, that whenever the balances change in the protocol, the ratio
between the two amounts should remain constant, hence the \texttt{k}.
However, this is broken due to the extra incentive in the
\texttt{\_swap} function. Meaning that over time the protocol funds will
be drained.

The follow block of code is responsible for the issue.

\begin{Shaded}
\begin{Highlighting}[]
\NormalTok{        swap\_count}\OperatorTok{++;}
        \ControlFlowTok{if}\NormalTok{ (swap\_count }\OperatorTok{\textgreater{}=}\NormalTok{ SWAP\_COUNT\_MAX) \{}
\NormalTok{            swap\_count }\OperatorTok{=} \DecValTok{0}\OperatorTok{;}
\NormalTok{            outputToken}\OperatorTok{.}\FunctionTok{safeTransfer}\NormalTok{(msg}\OperatorTok{.}\AttributeTok{sender}\OperatorTok{,} \DecValTok{1\_000\_000\_000\_000\_000\_000}\NormalTok{)}\OperatorTok{;}
\NormalTok{        \}}
\end{Highlighting}
\end{Shaded}

\textbf{Impact:} A user could maliciously drain the protocol of funds by
doing a lot of swaps and collecting the extra incentive given out by the
protocol.

Most simply put, the protocol's core invariant is broken.

\textbf{Proof of Concept:} 1. A user swaps 10 times, and collects the
extra incentive of \texttt{1\_000\_000\_000\_000\_000\_000} tokens 2.
That user continues to swap untill all the protocol funds are drained

Proof Of Code

Place the following into \texttt{TSwapPool.t.sol}.

\begin{Shaded}
\begin{Highlighting}[]

    \KeywordTok{function} \FunctionTok{testInvariantBroken}\NormalTok{() }\KeywordTok{public}\NormalTok{ \{}
\NormalTok{        vm}\OperatorTok{.}\FunctionTok{startPrank}\NormalTok{(liquidityProvider)}\OperatorTok{;}
\NormalTok{        weth}\OperatorTok{.}\FunctionTok{approve}\NormalTok{(}\FunctionTok{address}\NormalTok{(pool)}\OperatorTok{,} \FloatTok{100e18}\NormalTok{)}\OperatorTok{;}
\NormalTok{        poolToken}\OperatorTok{.}\FunctionTok{approve}\NormalTok{(}\FunctionTok{address}\NormalTok{(pool)}\OperatorTok{,} \FloatTok{100e18}\NormalTok{)}\OperatorTok{;}
\NormalTok{        pool}\OperatorTok{.}\FunctionTok{deposit}\NormalTok{(}\FloatTok{100e18}\OperatorTok{,} \FloatTok{100e18}\OperatorTok{,} \FloatTok{100e18}\OperatorTok{,} \FunctionTok{uint64}\NormalTok{(block}\OperatorTok{.}\AttributeTok{timestamp}\NormalTok{))}\OperatorTok{;}
\NormalTok{        vm}\OperatorTok{.}\FunctionTok{stopPrank}\NormalTok{()}\OperatorTok{;}

\NormalTok{        uint256 outputWeth }\OperatorTok{=} \FloatTok{1e17}\OperatorTok{;}

\NormalTok{        vm}\OperatorTok{.}\FunctionTok{startPrank}\NormalTok{(user)}\OperatorTok{;}
\NormalTok{        poolToken}\OperatorTok{.}\FunctionTok{approve}\NormalTok{(}\FunctionTok{address}\NormalTok{(pool)}\OperatorTok{,} \FunctionTok{type}\NormalTok{(uint256)}\OperatorTok{.}\AttributeTok{max}\NormalTok{)}\OperatorTok{;}
\NormalTok{        poolToken}\OperatorTok{.}\FunctionTok{mint}\NormalTok{(user}\OperatorTok{,} \FloatTok{100e18}\NormalTok{)}\OperatorTok{;}
\NormalTok{        pool}\OperatorTok{.}\FunctionTok{swapExactOutput}\NormalTok{(poolToken}\OperatorTok{,}\NormalTok{ weth}\OperatorTok{,}\NormalTok{ outputWeth}\OperatorTok{,} \FunctionTok{uint64}\NormalTok{(block}\OperatorTok{.}\AttributeTok{timestamp}\NormalTok{))}\OperatorTok{;}
\NormalTok{        pool}\OperatorTok{.}\FunctionTok{swapExactOutput}\NormalTok{(poolToken}\OperatorTok{,}\NormalTok{ weth}\OperatorTok{,}\NormalTok{ outputWeth}\OperatorTok{,} \FunctionTok{uint64}\NormalTok{(block}\OperatorTok{.}\AttributeTok{timestamp}\NormalTok{))}\OperatorTok{;}
\NormalTok{        pool}\OperatorTok{.}\FunctionTok{swapExactOutput}\NormalTok{(poolToken}\OperatorTok{,}\NormalTok{ weth}\OperatorTok{,}\NormalTok{ outputWeth}\OperatorTok{,} \FunctionTok{uint64}\NormalTok{(block}\OperatorTok{.}\AttributeTok{timestamp}\NormalTok{))}\OperatorTok{;}
\NormalTok{        pool}\OperatorTok{.}\FunctionTok{swapExactOutput}\NormalTok{(poolToken}\OperatorTok{,}\NormalTok{ weth}\OperatorTok{,}\NormalTok{ outputWeth}\OperatorTok{,} \FunctionTok{uint64}\NormalTok{(block}\OperatorTok{.}\AttributeTok{timestamp}\NormalTok{))}\OperatorTok{;}
\NormalTok{        pool}\OperatorTok{.}\FunctionTok{swapExactOutput}\NormalTok{(poolToken}\OperatorTok{,}\NormalTok{ weth}\OperatorTok{,}\NormalTok{ outputWeth}\OperatorTok{,} \FunctionTok{uint64}\NormalTok{(block}\OperatorTok{.}\AttributeTok{timestamp}\NormalTok{))}\OperatorTok{;}
\NormalTok{        pool}\OperatorTok{.}\FunctionTok{swapExactOutput}\NormalTok{(poolToken}\OperatorTok{,}\NormalTok{ weth}\OperatorTok{,}\NormalTok{ outputWeth}\OperatorTok{,} \FunctionTok{uint64}\NormalTok{(block}\OperatorTok{.}\AttributeTok{timestamp}\NormalTok{))}\OperatorTok{;}
\NormalTok{        pool}\OperatorTok{.}\FunctionTok{swapExactOutput}\NormalTok{(poolToken}\OperatorTok{,}\NormalTok{ weth}\OperatorTok{,}\NormalTok{ outputWeth}\OperatorTok{,} \FunctionTok{uint64}\NormalTok{(block}\OperatorTok{.}\AttributeTok{timestamp}\NormalTok{))}\OperatorTok{;}
\NormalTok{        pool}\OperatorTok{.}\FunctionTok{swapExactOutput}\NormalTok{(poolToken}\OperatorTok{,}\NormalTok{ weth}\OperatorTok{,}\NormalTok{ outputWeth}\OperatorTok{,} \FunctionTok{uint64}\NormalTok{(block}\OperatorTok{.}\AttributeTok{timestamp}\NormalTok{))}\OperatorTok{;}
\NormalTok{        pool}\OperatorTok{.}\FunctionTok{swapExactOutput}\NormalTok{(poolToken}\OperatorTok{,}\NormalTok{ weth}\OperatorTok{,}\NormalTok{ outputWeth}\OperatorTok{,} \FunctionTok{uint64}\NormalTok{(block}\OperatorTok{.}\AttributeTok{timestamp}\NormalTok{))}\OperatorTok{;}

\NormalTok{        int256 startingY }\OperatorTok{=} \FunctionTok{int256}\NormalTok{(weth}\OperatorTok{.}\FunctionTok{balanceOf}\NormalTok{(}\FunctionTok{address}\NormalTok{(pool)))}\OperatorTok{;}
\NormalTok{        int256 expectedDeltaY }\OperatorTok{=} \FunctionTok{int256}\NormalTok{(}\OperatorTok{{-}}\DecValTok{1}\NormalTok{) }\OperatorTok{*} \FunctionTok{int256}\NormalTok{(outputWeth)}\OperatorTok{;}

\NormalTok{        pool}\OperatorTok{.}\FunctionTok{swapExactOutput}\NormalTok{(poolToken}\OperatorTok{,}\NormalTok{ weth}\OperatorTok{,}\NormalTok{ outputWeth}\OperatorTok{,} \FunctionTok{uint64}\NormalTok{(block}\OperatorTok{.}\AttributeTok{timestamp}\NormalTok{))}\OperatorTok{;}
\NormalTok{        vm}\OperatorTok{.}\FunctionTok{stopPrank}\NormalTok{()}\OperatorTok{;}

\NormalTok{        uint256 endingY }\OperatorTok{=}\NormalTok{ weth}\OperatorTok{.}\FunctionTok{balanceOf}\NormalTok{(}\FunctionTok{address}\NormalTok{(pool))}\OperatorTok{;}
\NormalTok{        int256 actualDeltaY }\OperatorTok{=} \FunctionTok{int256}\NormalTok{(endingY) }\OperatorTok{{-}} \FunctionTok{int256}\NormalTok{(startingY)}\OperatorTok{;}
        \FunctionTok{assertEq}\NormalTok{(actualDeltaY}\OperatorTok{,}\NormalTok{ expectedDeltaY)}\OperatorTok{;}
\NormalTok{    \}}
\end{Highlighting}
\end{Shaded}

\textbf{Recommended Mitigation:} Remove the extra incentive mechanism.
If you want to keep this in, we should account for the change in the x *
y = k protocol invariant. Or, we should set aside tokens in the same way
we do with fees.

\begin{Shaded}
\begin{Highlighting}[]
\StringTok{{-}        swap\_count++;}
\StringTok{{-}        // Fee{-}on{-}transfer}
\StringTok{{-}        if (swap\_count \textgreater{}= SWAP\_COUNT\_MAX) \{}
\StringTok{{-}            swap\_count = 0;}
\StringTok{{-}            outputToken.safeTransfer(msg.sender, 1\_000\_000\_000\_000\_000\_000);}
\StringTok{{-}        \}}
\end{Highlighting}
\end{Shaded}

\subsection{Low}\label{low}

\subsubsection{\texorpdfstring{{[}L-1{]}
\texttt{TSwapPool::LiquidityAdded} event has parameters out of
order}{{[}L-1{]} TSwapPool::LiquidityAdded event has parameters out of order}}\label{l-1-tswappoolliquidityadded-event-has-parameters-out-of-order}

\textbf{Description:} When the \texttt{LiquidityAdded} event is emitted
in the \texttt{TSwapPool::\_addLiquidityMintAndTransfer} function, it
logs values in an incorrect order. The \texttt{poolTokensToDeposit}
value should go in the third parameter position, whereas the
\texttt{wethToDeposit} value should go second.

\textbf{Impact:} Event emission is incorrect, leading to off-chain
functions potentially malfunctioning.

\textbf{Recommended Mitigation:}

\begin{Shaded}
\begin{Highlighting}[]
\StringTok{{-} emit LiquidityAdded(msg.sender, poolTokensToDeposit, wethToDeposit);}
\VariableTok{+ emit LiquidityAdded(msg.sender, wethToDeposit, poolTokensToDeposit);}
\end{Highlighting}
\end{Shaded}

\subsubsection{\texorpdfstring{{[}L-2{]} Default value returned by
\texttt{TSwapPool::swapExactInput} results in incorrect return value
given}{{[}L-2{]} Default value returned by TSwapPool::swapExactInput results in incorrect return value given}}\label{l-2-default-value-returned-by-tswappoolswapexactinput-results-in-incorrect-return-value-given}

\textbf{Description:} The \texttt{swapExactInput} function is expected
to return the actual amount of tokens bought by the caller. However,
while it declares the named return value \texttt{ouput} it is never
assigned a value, nor uses an explict return statement.

\textbf{Impact:} The return value will always be 0, giving incorrect
information to the caller.

\textbf{Recommended Mitigation:}

\begin{Shaded}
\begin{Highlighting}[]
\NormalTok{    \{}
\NormalTok{        uint256 inputReserves = inputToken.balanceOf(address(this));}
\NormalTok{        uint256 outputReserves = outputToken.balanceOf(address(this));}

\StringTok{{-}        uint256 outputAmount = getOutputAmountBasedOnInput(inputAmount, inputReserves, outputReserves);}
\VariableTok{+        output = getOutputAmountBasedOnInput(inputAmount, inputReserves, outputReserves);}

\StringTok{{-}        if (output \textless{} minOutputAmount) \{}
\StringTok{{-}            revert TSwapPool\_\_OutputTooLow(outputAmount, minOutputAmount);}
\VariableTok{+        if (output \textless{} minOutputAmount) \{}
\VariableTok{+            revert TSwapPool\_\_OutputTooLow(outputAmount, minOutputAmount);}
\NormalTok{        \}}

\StringTok{{-}        \_swap(inputToken, inputAmount, outputToken, outputAmount);}
\VariableTok{+        \_swap(inputToken, inputAmount, outputToken, output);}
\NormalTok{    \}}
\end{Highlighting}
\end{Shaded}

\subsection{Informationals}\label{informationals}

\subsubsection{\texorpdfstring{{[}I-1{]}
\texttt{PoolFactory::PoolFactory\_\_PoolDoesNotExist} is not used and
should be
removed}{{[}I-1{]} PoolFactory::PoolFactory\_\_PoolDoesNotExist is not used and should be removed}}\label{i-1-poolfactorypoolfactory__pooldoesnotexist-is-not-used-and-should-be-removed}

\begin{Shaded}
\begin{Highlighting}[]
\StringTok{{-} error PoolFactory\_\_PoolDoesNotExist(address tokenAddress);}
\end{Highlighting}
\end{Shaded}

\subsubsection{{[}I-2{]} Lacking zero address
checks}\label{i-2-lacking-zero-address-checks}

\begin{Shaded}
\begin{Highlighting}[]
\NormalTok{    constructor(address wethToken) \{}
\VariableTok{+       if(wethToken == address(0)) \{}
\VariableTok{+            revert();}
\VariableTok{+        \}}
\NormalTok{        i\_wethToken = wethToken;}
\NormalTok{    \}}
\end{Highlighting}
\end{Shaded}

\subsubsection{\texorpdfstring{{[}I-3{]}
\texttt{PoolFacotry::createPool} should use \texttt{.symbol()} instead
of
\texttt{.name()}}{{[}I-3{]} PoolFacotry::createPool should use .symbol() instead of .name()}}\label{i-3-poolfacotrycreatepool-should-use-.symbol-instead-of-.name}

\begin{Shaded}
\begin{Highlighting}[]
\StringTok{{-}        string memory liquidityTokenSymbol = string.concat("ts", IERC20(tokenAddress).name());}
\VariableTok{+        string memory liquidityTokenSymbol = string.concat("ts", IERC20(tokenAddress).symbol());}
\end{Highlighting}
\end{Shaded}

\subsubsection{\texorpdfstring{{[}I-4{]} Event is missing
\texttt{indexed}
fields}{{[}I-4{]} Event is missing indexed fields}}\label{i-4-event-is-missing-indexed-fields}

Index event fields make the field more quickly accessible to off-chain
tools that parse events. However, note that each index field costs extra
gas during emission, so it's not necessarily best to index the maximum
allowed per event (three fields). Each event should use three indexed
fields if there are three or more fields, and gas usage is not
particularly of concern for the events in question. If there are fewer
than three fields, all of the fields should be indexed.

\begin{itemize}
\tightlist
\item
  Found in src/TSwapPool.sol: Line: 44
\item
  Found in src/PoolFactory.sol: Line: 37
\item
  Found in src/TSwapPool.sol: Line: 46
\item
  Found in src/TSwapPool.sol: Line: 43
\end{itemize}
